%Events are pre-selected with \texttt{RPVLL} filters, and the selected events are reprocessed with a special setup so that large radius tracks and secondary vertices are reconstructed as described in Section~\ref{sec:data_preparation}.
In this section, the analysis level selections applied to events, leptons, and and vertices are described.

\subsection{Event selection}
\label{sec:event_selection}
%Events are selected by minimum requirements on the quality of events, completeness of luminosity blocks, and HLTs used in this analysis. The event selections are as follows.
Minimum requirements are placed on events based on the quality of events, completeness of the corresponding luminosity blocks, primary vertex, and HLTs used in this analysis. In addition, cosmic veto is applied to reject events with back-to-back muons. The event selection is described below.

\begin{itemize}
    \item \texttt{GoodRunsList} removes events from incomplete luminosity blocks.
    \item Event cleaning removes corrupted/bad events due to problems in TileCal, LAr noise bursts, SCT recovery, or TTC restarts~\cite{ElectronPhotonRun2}.
    \item Events are required to pass one of the HLTs listed in Table~\ref{table:triggers}.
    \item Events are required to have at least one primary vertex along the beam line ($z<200$ \si{\mm}).
    \item Events are rejected if there is a pair of leptons with $R_{\mathrm{CR}} < 0.01$ where $R_{\mathrm{CR}} = \sqrt{(\Delta \phi - \pi)^{2} + (\Sigma \eta)^{2}}$.
\end{itemize}

The event selection is summarized in Table~\ref{table:signal_selection}.

\subsection{Muon and electron requirements}
\label{sec:muon_electron_selection}
Prior to applying the vertex level selections, tracks from vertices are required pass the requirements on kinematics, overlap removal, and muon and electron identification criteria.

Electron requirements are based on the recommendations from EGamma group~\cite{ElectronPhotonRun2} with a few optimization for electrons with large impact parameters. Electrons are rejected if there is a bad cluster associated with an electron. Basic kinematic cuts are applied to electrons, $|\eta| < 2.47$ and $p_{T}$ > 7 GeV. The electron \texttt{LooseLH} working point is used, but the requirements on $d_{0}$ and Pixel hits are removed to improve electron detection efficiency at large impact parameters.

Muon requirements are based on the recommendations from MuonCombinedPerformance group~\cite{MuonRun2}. Muon \texttt{Loose} working point is used for the identification criteria, and a fiducial cut, $|\eta| < 2.5$, and kinematic cut, $p_{T}$ > 10 GeV, are applied to muons. The requirements on Pixel hits are removed to improve muon detection efficiency at large impact parameters. In addition, muons are required to be \texttt{CombinedMuons} to ensure that they have associated ID tracks for vertex reconstruction. In case of MC samples, muon momentum resolution and scale correction are applied to the simulated muons for better agreement between data and simulation~\cite{Aad:2016jkr}. 

Overlap removal is applied to both muons and electrons to ensure that a ID track is associated with only one muon or electron.%Muons or electrons are removed from vertices if they do not pass the muon or electron requirements.
The muon and electron requirements are summarized in Table~\ref{table:lepton_requirement}.

\begin{table}[!htb]
  \centering
  \begin{tabular}{ l  l }
    \hline
    \hline
    \textbf{Muon}     &       Overlap removal                                                             \\
                                  &       Muon \texttt{Loose}                                                         \\
                                  &       $|\eta| < 2.5$                                                           \\
                                  &       $p_{T} > 10.0$ GeV                                                       \\
                                  &       Combined Muon                                                            \\
    \hline
    \textbf{Electron} &       Overlap removal                                                             \\
                                  &       Bad cluster removal                                                         \\
                                  &       Electron \texttt{LooseLH} (no requirement related to $d_{0}$, Pixel hits)   \\
                                  &       $|\eta| < 2.47$                                                             \\
                                  &       $p_{T} > 7.0$ GeV                                                           \\
    \hline
    \hline
  \end{tabular}
  \caption{Muon and electron requirements applied at analysis level.}
  \label{table:lepton_requirement}
\end{table}

\subsection{Vertex selection}
\label{sec:vertex_selection}
The vertex selection is applied to two-track secondary vertices found in Section~\ref{sec:secondary_vertex_reconstruction}. Secondary vertices with minimum displacement of 2 mm from the primary vertex are selected. The selected displaced vertices are made of two tracks which can be any combination of muon, electron, and non-lepton tracks. Therefore, vertices are separated into three vertex types, control, validation, and signal regions.

In the control region, vertices are required to have two non-leptonic tracks (xx). In the validation region, vertices are required to have a muon or an electron and another non-leptonic track ($\mu$x, $e$x). In signal region, vertices are required to have a muon pair, an electron pair, or a muon-electron pair ($\mu\mu$, $ee$, $e\mu$). The control region and the validation region are used for background (Section~\ref{sec:background_estimation}) and systematic uncertainty (Section~\ref{sec:syst}) estimations. The control, validation, and signal regions are summarized in Table~\ref{table:vertex_type}.

\begin{table}[!h]
  \centering
  \begin{tabular}{ l  c}
    \hline
    \hline
	Region				& Vertex Type										\\
    \hline
	Control     		& xx $xx$   										\\
	Validation       	& $\mu$x, $e$x										\\
	Signal       		& $\mu\mu$, $ee$, $e\mu$							\\
    \hline
    \hline
  \end{tabular}
  \caption{The control, validation, and signal regions defined by the vertex type.}
  \label{table:vertex_type}
\end{table}

In all regions, vertices are required to pass a common set of vertex selections described as follows. Vertices are required to have $\chi^2 / \mathrm{ DOF} <$ 5 to reject poorly reconstructed vertices. A minimum transverse displacement of 2 mm from the primary vertex is required to suppress background from prompt decays. Two tracks from a vertex are required to have opposite charges. Vertices are rejected if they are within the volume of disabled Pixel module~\cite{Backhaus:2110260}. Hadronic interaction of charged particles with detector material is a major source of backgrounds. Therefore, the vertices are rejected if they are within dense detector material~\cite{Aaboud:2016poq}. The material veto is not applied to $\mu\mu$ type vertex due to low probability of muon interaction with detector material. Vertices are also required to be in the detector volume covered by the material mapping ($r < 300$ \si{mm}, $z < 300$ \si{mm}). The vertices are required to have $m >$ 10 GeV to suppress backgrounds from low mass SM particles such as $J/\Psi$. The vertex mass is calculated by the secondary vertex reconstruction algorithm with the assumption that all tracks have pion mass. Cosmic veto is applied to vertices by requiring $R_{\mathrm{CR}} > 0.01$. The cosmic veto is very effective in rejecting cosmic muons reconstructed as back-to-back muon vertices, and the details are discussed in Section~\ref{sec:cosmic_ray}

In addition to the common vertex selection, at least one electron or muon from the vertex is required to be matched with one of the triggers listed on Table~\ref{table:triggers} and the filters listed on Table~\ref{table:rpvll_filter_selection} in the signal region. The vertex selection is summarized in Table~\ref{table:signal_selection}.

\begin{table}[!htb]
  \centering
  \begin{tabular}{ l  l }
    \hline
    \hline
    \textbf{Event}           &       GoodRunsList (Section~\ref{sec:preselection})                               \\
                             &       Trigger filter (Section~\ref{sec:preselection})                             \\
                             &       Event cleaning (Section~\ref{sec:event_selection})                          \\
                             &       Cosmic veto    (Section~\ref{sec:event_selection})                          \\
                             &       $z_{\mathrm{PV}} < 200$ mm (Section~\ref{sec:event_selection})                       \\
    \hline
    \textbf{Vertex}&       Trigger matching (signal region only)                                          \\
                             &       $\chi^2 / \mathrm{ DOF} < 5$                                                \\
                             &       $r > $2 mm                                                           \\
                             &       Opposite charge                                                             \\
                             &       Disabled module veto                                                        \\
                             &       Material veto (excluding $\mu\mu$)                                                              \\
                             &       $m > 10$ GeV                                                                \\
                             &       $R_{\mathrm{CR}} > 0.01$                                                    \\
                             &       $r < 300$ \si{mm}, $z < 300$ \si{mm}                                        \\
                             &       Filter matching (signal region only)                                                            \\
    \hline
    \hline
  \end{tabular}
  \caption{Event and vertex selections applied to select displaced vertices.}
  \label{table:signal_selection}
\end{table}


