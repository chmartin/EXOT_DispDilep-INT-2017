%This displaced vertex signature has the potential to provide an evidence for Physics beyond the SM, and it provides very clean signature with minimum backgrounds from the SM processes.
The search for long-lived particles (LLP) is an important part of the program in searching for new physics at the LHC. Many extensions to the Standard Model (SM) such as split SUSY~\cite{Hewett:2004nw}~\cite{ArkaniHamed:2004yi}, MSSM with R-parity violation~\cite{Barbier:2004ez}, or Hidden Valley~\cite{Han:2007ae} predict the production of neutral, weakly-coupled particles with long lifetimes compatible with the dimension of the ATLAS detector. In particular, several theories, including R-parity violation, Hidden valley, or $Z'$ models with long-lived neutrinos~\cite{Basso:2008iv}, predict the existence of LLPs that can decay to final-states containing a displaced vertex with a pair of leptons.

This paper presents the search for a heavy, long-lived neutral particle decaying to a dilepton pair, $\mu\mu$, $ee$, or $e\mu$ within the ATLAS Inner Detector (ID). The LLP is referred as $Z'$ but with no assumption on $Z'$ production mechanism for a model-independent search. For the purpose of establishing a signal benchmark, the LLP is singly produced in Drell-Yan process with $Z'$ mass ranges from 100 GeV to 1 TeV and $c\tau$ between 100 mm and 500 mm. The displaced vertex provides a clean signature with minimum backgrounds from the SM processes.

There have been several searches for the LLPs produced in $pp$ collisions in Run I at $\sqrt{s} =$ 8 TeV, including the search for displaced hadronic jet~\cite{Blackburn:1550730}, displaced heavy flavors~\cite{Harris:1512932}, or multi-track displaced vertex~\cite{Aad:2015rba}, and no significant excess was observed. This paper presents the search for a different signature, and it is one of the first efforts\footnote{SUSY displaced dilepton search in Run II is looking for a displaced dilepton signature in the context of supersymmetric models.} in the ATLAS experiment to search for a genetic displaced vertex signature decaying to a dilepton pair.

This analysis uses 32.8 $\mathrm{fb^{-1}}$ of $pp$ collision data at $\sqrt{s}=13$ TeV collected in 2016 using the ATLAS detector. In order to gain sensitivity for the non-conventional signature of LLPs, a special setup of data reprocessing and reconstruction, called \textit{Large radius tracking}, is used. This setup is described in ~\ref{sec:track_vertex_reconstruction}. The special setup allows the reconstruction of tracks with large impact parameters and secondary vertices significantly displaced from primary vertices.

The analysis shares the technical setup and the analysis framework~\cite{SUSYAnalysisFramework} with SUSY displaced multi-track vertex search~\cite{Duarte-Campderros:2152010} and SUSY displaced dilepton search~\cite{}. The former searches for the LLPs decaying to displaced vertices with high track multiplicity and large missing energy, and the latter looks for the same displaced dilepton signature as this search but in the context of supersymmetric models. 

%This analysis focuses on interpreting the displaced dilepton from a LLP decay in the context of model-independent, Exotics search.
This analysis focuses on interpreting the LLPs decaying to displaced dilepton vertices in the context of model-independent, exotic resonance search.

